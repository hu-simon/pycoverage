\documentclass{article}
\usepackage{ps}
\usepackage{comment}
\usepackage{algorithm}
\usepackage{algorithmic}
\usepackage{etoolbox}
\AtBeginEnvironment{algorithm}{%
	\setlength{\columnwidth}{\linewidth}%
}

% Glossary entries.
\renewcommand*{\arraystretch}{1}% default is 1
\newcommand{\glsdescwidth}{13cm}
\usepackage[sort=standard,savewrites, nonumberlist, seeautonumberlist]{glossaries}
\newglossary[slg]{symbolslist}{syi}{syg}{List of Symbols}
\newglossarystyle{mystyle}{%
	\glossarystyle{long}%
	\renewenvironment{theglossary}%
	{\begin{longtable}{p{3cm}p{\glsdescwidth}}}%
		{\end{longtable}}%
}
\makeglossaries

% New glossary entries.
\newglossaryentry{symb:Xi}{name=\ensuremath{\Xi},
	description={Domain of operation. Assumed to be a subset of $\R^n$, and is compact and convex},
	sort={b},
	type=symbolslist}
\newglossaryentry{symb:xi}{name=\ensuremath{\xi},
	description={Variable used to represent a member of $\Xi$. Also a spatial integration variable},
	sort={c},
	type=symbolslist}
\newglossaryentry{symb:t}{name=\ensuremath{t},
	description={Variable used to represent time and is assumed to be in $\R^+$. Also a temporal integration variable},
	sort={d},
	type=symbolslist}
\newglossaryentry{symb:phi}{name=\ensuremath{\varphi},
	description={Function that encodes the area(s) where we want effective coverage. Assumed to be integrable in the Lesbegue sense},
	sort={e},
	type=symbolslist}
\newglossaryentry{p}{name=\ensuremath{p},
	description={Vector containing position of Agents. $p = (p_1, p_2, \dots, p_n)$},
	sort={f},
	type=symbolslist}
\newglossaryentry{N}{name=\ensuremath{N},
	description={Number of BLUE Agents},
	sort={a},
	type=symbolslist}
\newglossaryentry{XiN}{name=\ensuremath{\Xi^N},
	description={$N$-th dimensional Cartesian product of $\Xi$},
	sort={g},
	type=symbolslist}
\newglossaryentry{Vori}{name=\ensuremath{\V(\Xi, p)},
	description={Voronoi partition of $\Xi$ given $p$},
	sort={h},
	type=symbolslist}
\newglossaryentry{voricell}{name=\ensuremath{\V_i},
	description={$i$-th Voronoi cell},
	sort={i},
	type=symbolslist}
\newglossaryentry{delta}{name=\ensuremath{\delta},
	description={Some measure of the distance between a RED Agent and a BLUE Agent},
	sort={j},
	type=symbolslist}
\newglossaryentry{mi}{name=\ensuremath{M_i},
	description={Mass of $i$-th Voronoi cell},
	sort={k},
	type=symbolslist}
\newglossaryentry{ci}{name=\ensuremath{C_i},
	description={Center of mass of $i$-th Voronoi cell},
	sort={l},
	type=symbolslist}

\DeclareMathOperator{\bxi}{\boldsymbol{\xi}}

% TItle information
\title{\bf\Large Coverage Control Overview}
\author{\bf Simon Hu}
\date{}

\begin{document}

\maketitle 
\glsaddall
\printglossary[style=mystyle, type=symbolslist]
\newpage

\section{Coverage Control Overview}
\subsection{Background}
Let $\bp = (p_1, p_2, \dots, p_N) \in \R^N$ be a vector whose elements are the positions of the $i$-th agent, given by $p_i \in \R^d$ where $d=1, 2, \dots$. The goal of the coverage control algorithm is to solve the following problem.
\begin{equation}
	\label{eq:problem statement}
	\displaystyle \max\limits_{\bp \in \Xi^N}{\mathcal{H}_{\varphi}(\bp, t)} := \max\limits_{\bp \in \Xi^N}{\int_{\Xi}{\min\limits_{i=1, 2, \dots, N}{\norm{\bxi - \bp_i}^2_2} \: \varphi(\bxi, t) \: \dif \bxi}}
\end{equation}
In other words, the goal is to find the optimal configuration of agent positions $\bp$ so that the desired area to cover, which is encoded in $\varphi : \R^N \times \R^+ \to \R$, is covered by all agents, which ensuring that agents are assigned an area, which is encoded in $\norm{\bxi - \bp}_2^2$, that is maximal with respect to $\Xi$. 

The Voronoi partition $\mathcal{V} \equiv \mathcal{V}(\Xi, \bp)$ of $\Xi$ given the current agent positions $\bp$ is a set $\mathcal{V} = \left\{ \mathcal{V}_1, \mathcal{V}_2, \dots, \mathcal{V}_N \right\}$. Here, each cell $\mathcal{V}_i$ is given by 
\begin{equation*}
	\displaystyle \mathcal{V}_i = \left\{ \xi \: | \: \norm{\xi - p_i}^2_2 \leq \norm{\xi - p_j}^2_2 \: \forall \: i \neq j, \: \forall \: j = 1, 2, \dots, n \right\}.
\end{equation*}
An example of a Voronoi partition is shown in Figure (insert figure reference here). Using the definition of a Voronoi partition, we can rewrite (\ref{eq:problem statement}) as 
\begin{equation}
	\label{eq:rewritten coverage control}
	\displaystyle \max\limits_{\bp \in \Xi^N}{\mathcal{H}_{\varphi}(\bp, t)} := \max\limits_{p \in \Xi^N}\sum\limits_{i=1}^{N}{\int_{\mathcal{V}_i}{\norm{\bxi - \bp_i}^2_2 \varphi(\bxi, t)\: \dif \bxi}}.
\end{equation}
In other words, the problem reduces to finding the configuration of agents so that effective coverage is maintained, but each agent is also assigned an area (i.e. a cell) that it is responsible for. There are two advantages of using the Voronoi partitions. First, the $\min$ term inside the integral is removed by the construction of the Voronoi cells. Second, any algorithm that uses these partitions will be \textit{distributed}, which means agents only need to use information from its Voronoi neighbors, defined by agents that share a cell boundary (i.e. agents $i$ and $j$ are neighbors if and only if $\partial \mathcal{V}_i \cap \partial \mathcal{V}_j \neq \emptyset$), so that communications between agents can be reduced. 

The mass $M_i$ and centroid $C_i$ of the $i$-th cell is given by 
\begin{equation}
	\label{eq:mass, center of mass}
	\displaystyle M_i = \int_{\mathcal{V}_i}{\varphi(\bxi, t)\: \dif \bxi}, \:\:\:\: C_i = \frac{1}{M_i}\int_{\mathcal{V}_i}{\bxi \varphi(\bxi, t) \: \dif \bxi}.
\end{equation}
To solve the maximization problem, agents move towards the centroid of their Voronoi cell \cite{Cortes:2004}. Intuitively, the agents move towards the area they need to cover since the $\varphi$ acts as an attracting force that draws the center of mass towards the desired coverage area. Algorithm (insert algorithm reference here) describes one step of the coverage control algorithm. 
\subsection{Approach}
For the purpose of this project, we consider $d = 3$, i.e. 3-D Euclidean space. Though orientation information about the robot may be present, so that $d \neq 3$, we only use the position information as our output is waypoints the agents should travel to.
\end{document}